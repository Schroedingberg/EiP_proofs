% Created 2017-02-18 Sa 16:25
% Intended LaTeX compiler: pdflatex
\documentclass[11pt]{article}
\usepackage[utf8]{inputenc}
\usepackage[T1]{fontenc}
\usepackage{graphicx}
\usepackage{grffile}
\usepackage{longtable}
\usepackage{wrapfig}
\usepackage{rotating}
\usepackage[normalem]{ulem}
\usepackage{amsmath}
\usepackage{textcomp}
\usepackage{amssymb}
\usepackage{capt-of}
\usepackage{hyperref}
\usepackage[english]{babel}
\author{Aaron}
\date{\today}
\title{}
\hypersetup{
 pdfauthor={Aaron},
 pdftitle={},
 pdfkeywords={},
 pdfsubject={},
 pdfcreator={Emacs 25.1.1 (Org mode 9.0.5)}, 
 pdflang={English}}
\begin{document}

\section{Sheet 12, Aufgabe 2}
\label{sec:org75b4ba5}
\begin{verbatim}
def recursive_method(n):
    if n==0:
        return 1
    else:
        return recursive_method(n-1)+2**n
\end{verbatim}

\begin{description}
\item[{Behauptung}] Die Funktion berechnet für jede natürliche Zahl n den Wert \(2^{n+1}-1\)
\item[{Umschreiben als rekursive Gleichung}] \(R_n = R_{n-1}+2^n, n \geq 1 ,R_0 = 1\)
\item[{Behauptung}] \(A(n) : R_n = 2^{n+1} -1\)
\item Beweis durch vollständige Induktion nach n
\begin{description}
\item[{Anfang}] \(A(1)\) wahr.
Beweis: \(2^{1+1} -1 = R_{1-1} + 2^1 \Leftrightarrow 2^{2}-1 = 1+2 \Rightarrow A(1) \text{wahr}\)
\item[{Annahme (IA)}] \(A(n)\) wahr für ein festes \(n\in \mathbb{N}\)
\item[{Schritt}] Zeige \(A(n+1)\) wahr
Behauptung: \(R_{n+1} = 2^{n+2}-1\)
Beweis: 
\begin{eqnarray*}
R_{n+1} & = R_n + 2^{n+1}\\
	& = 2^{n+1} -1 + 2^{n+1}\\
	& = 2\cdot 2^{n+1} -1 \\
	& = 2^{n+2} -1
\end{eqnarray*} QED
\end{description}
\end{description}
\end{document}